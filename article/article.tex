\documentclass[12pt]{article}

\usepackage{setspace}
\onehalfspacing

\usepackage{natbib}
\bibliographystyle{apalike}

\usepackage[margin=1.0in]{geometry}

\usepackage{amsmath}

\usepackage{lineno}

\usepackage{graphicx}
\linenumbers

%opening
\title{Predicting detection probabilities for rare and undersampled North American landbirds}
\author{
	Edwards, Brandon P.M.\\
	\and
	Solymos, Peter\\
	\and
	Stralberg, Diana\\
	\and
	Michel, Nicole\\
	\and
	Robinson, Barry\\
	\and
	Gahbauer, Marcel\\
	\and
	Grinde, Alexis\\
	\and
	Hope, David\\
	\and
	Bennett, Joseph R.\\
	\and
	Smith, Adam C.\\
}

\begin{document}

\maketitle

\begin{abstract}

\end{abstract}

\section{Introduction}

\subsection{Predictions}
\par We predict that for both the Bayesian hierarchical QPAD removal model and Bayesian hierarchical QPAD distance model, estimates of cue rate and EDR, respectively, will change very little from those estimated from the original NA-POPS database \cite{edwards_point_2023}. 
This is because the species that were modelled in \citet{edwards_point_2023} had a sufficient number of data points to stand on their own, and so adding in species-level effects should have little effect on estimates. 
We also predict that a multi-species model will increase the precision of the estimates of cue rate and EDR, but only up to a certain sample size. 
This is because there are many species for which NA-POPS has thousands to even tens of thousands of data points; given precision generally increases with sample size, after a certain sample size, a single-species vs multi-species estimate of cue rate and EDR should be similarly precise. 
Put differently, we expect that species with a smaller sample size will have precision of cue rate and EDR estimates improved with a multi-species model.
Finally, we expect that a multi-species model will be able to provide predictions of cue rate and EDR for species with little to no data, but should be able to provide better/more precise estimates for data-sparse species compared to data-absent species.
Additionally, we expect to see a higher precision of estimates for species where there are more "similar" species.
That is, species with more closely-related species modelled in the multi-species model will have more precise estimates of cue rate, and species with more species with similar traits will have more precise estimates of EDR.

\section{Methods}

\par The following subsections outline the modelling techniques used in this study as well as the data that were collected. 
We start by introducing the Bayesian hierarchical QPAD removal and distance models, and then describe the input.

\subsection{Data Collection}
\subsubsection{Point Count Data}
\par We used the same standardized database of point count data that was put together through the NA-POPS project \cite{edwards_point_2023}. 
This consists of $712138$ point counts conducted across $292$ projects in Canada and the United States, $422514$ of which used removal sampling and $522820$ of which used distance sampling.

\subsubsection{Phylogenetic Trees}
\par We accessed 2000 pseudo-posterior trees from \citet{jetz_global_2012} that contains phylogeny for all species that contained sufficient data for removal modelling ($>75$ observations as per \cite{edwards_point_2023,solymos_evaluating_2018,buckland_introduction_2001}). 
We also requested that the phylogenies include the following bird species with insufficient data from NA-POPS: LeConte's Thrasher, Bicknell's Thrush, Lesser Prairie-chicken, Harris's Sparrow, Kirtland's Warbler, Tricolored Blackbird, and Spotted Owl. 
All seven of these bird species are listed on Partner's in Flight's watchlist for various threats (CITATION). 
This resulted in 2000 pseudo-posterior trees that contained 323 species.

\par We generated a phylogenetic correlation matrix using the same methods outlined in the Appendix of \citet{solymos_phylogeny_2018} that generate distances between species, and further processed using the R package "ape" \citet{paradis_ape_2019}.

\subsubsection{Species Traits}
\par Species traits (song pitch, migratory vs resident, open vs. closed habitat) were obtained from Birds of the World accounts of all the birds to model in this study (CITATION). 
Body mass for each bird species was obtained form the Elton traits dataset (CITATION). 
Where body mass was not available through the Elton traits dataset, the Birds of the World account was used instead.

\subsection{Bayesian Hierarchical Models}

\par For any point count that employs removal and/or distance sampling, let $Y_{sijk}$ be the observed count of species $s$ during sampling event $i$, occuring in time band $j \in [1,J]$ and/or distance band $k \in [1,K]$.
That is, if we have a point count where removal sampling or distance sampling (but not both) is not used, then we have that either $J = 1$ or $K = 1$, respectively. 
The following two subsections detail the Bayesian hierarchical QPAD Removal and Distance Models.

\subsubsection{Bayesian Hierarchical QPAD Removal Model}

\par For removal modelling, we are not interested in distance to the bird, and so we will sum counts from each sampling event $i$ over all distance bands. 
Thus, we will be considering counts $Y_{sij.}$. 
From \citet{solymos_calibrating_2013}, we model $Y_{sij.}$ as
$$Y_{sij.} \sim multinomial\left(Y_{si..}, \Pi_{si.}\right).$$

\par For sample $i$, $\Pi_{si.}$ is the corresponding probability vector that determines how the total number of species $s$ observed at sample $i$ (i.e., $Y_{si..}$) are to be distributed across the $J$ time bands; in essense, it is the mixing parameter.
Let $t_ij$ be the maximum time for time band $j$ during sampling event $i$, and let $phi_s$ be the unknown cue rate for species $s$.
We then have the following calculation of each component $\pi_{sij}$ of the mixing parameter $\Pi_{si.}$:

\begin{equation*}
	\pi_{sij} = 
	\begin{cases}
		\dfrac{\exp\left\{ -t_{i,j-1}\phi_{s} \right\} - \exp\left\{ -t_{ij}\phi_{s} \right\}}{1 - \exp\left\{ -t_{iJ}\phi_{s} \right\}} & \text{for } j > 1 \\
		1 - \sum_{n = 2}^{J} \pi_{sin} & \text{for } j = 1
	\end{cases}
\end{equation*}

\par We are most interested in estimating $phi_s$, the cue rate for species $s$. 
To allow for the partial pooling of estimates for each species to be based on a phylogenic relationship, we can use Gaussian processes, which are effectively allow for "continuous" factors. 
Let $\Phi$ be the vector of cue rates (of length $s$). 
We then have:

$$\log \Phi \sim MVN\left( \mu_{[s]}, \Sigma_{[s \times s]} \right).$$

\par The covariance function $\Sigma$ (denoted with subscript $[s \times s]$ to show dimensionality) determines the strength of the partial pooling, and can be any function that contains a measure of distance. 
In this case, we created a function based on phylogenetic distance $D_{[s \times s]}$, i.e.,

$$\Sigma_{[s \times s]} = D_{[s \times s]}\lambda\sigma.$$

\par Here, $\lambda$ is Pagal's Lambda (CITATION), which determines the strength of the phylogenetic relationship. 
We set $\lambda = 0.76$, consistent with previous results \cite{solymos_phylogeny_2018}. 
We also have the extra variance term $\sigma \sim exp(5)$.

\par Finally, the vector of mean cue rate $\mu_{[s]}$ is just a function of the species migratory strategy $\alpha_M$, where the migratory strategy $M$ could either be Resident or Migrant. 
We set priors on both migratory strategies to be $\alpha_M \sim N(-1, 0.01)$, because on the log scale we would expect cue rate to be negative. 

\par Results of simulated priors can be found in Appendix TO DO.

\subsubsection{Bayesian Hierarchical QPAD Distance Model}

\par For distance modelling, we are not interested in the time bin that the bird was recorded in, and so we will sum counts from each sampling event $i$ over all time bands. 
Thus, we will be considering counts $Y_{si.k}$. 
From \citet{solymos_calibrating_2013}, we model $Y_{si.k}$ as
$$Y_{si.k} \sim multinomial\left(Y_{si..}, \Pi_{si.}\right).$$

\par For sample $i$, $\Pi_{si.}$ is the corresponding probability vector that determines how the total number of species $s$ observed at sample $i$ (i.e., $Y_{si..}$) are to be distributed across the $K$ distance bins; in essense, it is the mixing parameter. 
Let $r_ik$ be the maximum distance for distance bin $k$ during sampling event $i$, and let $tau_s$ be the unknown effective detection radius for species $s$. 
We then have the following calculation of each component $\pi_{sik}$ of the mixing parameter $\Pi_{si.}$:

%\begin{equation*}
%	\pi_{sik} = 
%	\begin{cases}
%		\dfrac{\left( 1 - \exp\left\{ -\dfrac{r_{i,k}^2}{\exp\left\{\tau_s^2\right\}} \right\} \right) - \left( 1 - \exp\left\{ -\dfrac{r_{i,k-1}^2}{\exp\left\{\tau_s^2\right\}} \right\} \right)}
%		{ 1 - \exp\left\{ -\dfrac{r_{i,K}^2}{\exp\left\{\tau_s^2\right\}} \right\} } & \text{for } k > 1 \\
%		1 - \sum_{n = 2}^{J} \pi_{sin} & \text{for } k = 1
%	\end{cases}
%\end{equation*}

\begin{equation*}
	\pi_{sik} = 
	\begin{cases}
		\dfrac{f(r_{i,k}, \tau_s) - f(r_{i,k-1}, \tau_s)}{f(r_{i,K}, \tau_s)} & \text{for } k > 1 \\
		1 - \sum_{n = 2}^{J} \pi_{sin} & \text{for } k = 1
	\end{cases}
\end{equation*}

where 
$$f(r,\tau) =  1 - \exp\left\{ -\dfrac{r^2}{\exp\left\{\tau^2\right\}} \right\} .$$

\par We are most interested in estimating $\tau_s$, the effective detection radius for species $s$. 
To allow for the partial pooling of estimates for each species to be based on traits, we set up a mixed effects model as follows:
$$\log \tau_s \sim N(\mu_s, \sigma).$$

\par In this model, mean effective detection radius for species $s$ (i.e., $\mu_s$) is a function of the following traits as described in \cite{solymos_phylogeny_2018}: an overall intercept term $\alpha_0$; migratory strategy $\alpha_M$, where $M$ can be Migratory or Resident; habitat preference $\alpha_H$, where $H$ can be Open or Closed habitat; log body size $\beta_b$; and log song pitch $\beta_p$. 
That is, we have
$$ \mu_s = \alpha_0 + \alpha_{M_s} + \alpha_{H_s} + \beta_b \times \log BodySize_s + \beta_p \times \log SongPitch_s$$
$$\alpha_0 \sim N(0.05, 0.1)$$
$$ \alpha_M, \alpha_H \sim N(0, 0.05)$$
$$ \beta_b \sim N(0.01, 0.005)$$
$$ \beta_p \sim N(-0.01, 0.005) $$
$$\sigma \sim exp(5)$$

\par Results of simulated priors can be found in Appendix TO DO.

\subsection{Model Assessment}

\par For both the Bayesian hierarchical QPAD Removal and Distance models described above, we compared the estimated cue rates and effective detection radii, respectively, to those produced within a single-species context in \citet{edwards_point_2023}. 
To do this, we modelled the difference as
$$\Delta X \sim N(0,\sigma)$$
$$\sigma \sim exp(1)$$

where $\Delta X$ is calculated as the mean multi-species estimate of cue rate or EDR minus the mean single-species estimate of cue rate of EDR.
If our predictions are true, we would expect to see a distribution of modelled differences centred about 0, indicating no real difference in estimates of cue rate or EDR between single-species and multi-species models, for species for which we have sufficient data.

\par We were also interested in determining if a multi-species model improved the precision around the estimates of cue rate or EDR. 
To do this, we modelled the standard deviations about each estimate of cue rate or EDR from both the multi-species and single-species models from \citet{edwards_point_2023}, and regressed them against the log sample size per species.
That is
$$SD \sim N(\alpha_M + \beta_M\times \log(N), \sigma)$$
$$\alpha, \beta \sim N(0,1)$$
$$\sigma \sim exp(1)$$

where $\alpha$ is the model-specific intercept and $\beta$ is the model-specific slope.

\par If our predictions are true, we would expect to see a higher value of $\alpha_1$ than for $\alpha_2$, corresponding to a higher baseline standard deviation in the single species models.
We would also likely see a slightly smaller (i.e., more negative) value of $\beta_1$ than $\beta_2$ (which should also be negative), indicating that the standard deviation decreases with log sample size for both models, but should decrease at a slower rate with the multi-species model (which has a lower baseline standard deviation to begin with) than the single-species model.

\subsection{Predicting Detection Probabilities}
\par To predict estimates of cue rate or EDR from the multi-species models, we adopted two possible approaches.
If the NA-POPS database contained some point-count data for the species of interest, we simply added these data into the model.
This is different from \citet{edwards_point_2023, solymos_calibrating_2013, solymos_evaluating_2018}, where any species with less than 75 data points were removed from the analysis.
For species that did not have any data in the NA-POPS database, we generated a row of traits data for the species and modelled the species just as a parameter.


\section{Results}
\includegraphics{../output/plots/removal_1vs1.png}

\includegraphics{../output/plots/removal_sd_plot.png}

\includegraphics{../output/plots/removal_predictions_plot.png}

\includegraphics{../output/plots/distance_1vs1.png}

\includegraphics{../output/plots/distance_sd_plot.png}

\includegraphics{../output/plots/distance_predictions_plot.png}

\bibliography{refs}

\end{document}

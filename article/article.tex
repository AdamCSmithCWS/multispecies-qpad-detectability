\documentclass[]{article}

%opening
\title{Predicting detection probabilities for rare and undersampled birds}
\author{}

\begin{document}

\maketitle

\begin{abstract}
	
	Insert abstract here.

\end{abstract}

\section{Introduction}

Detection probability, or detectability, is an important metric to consider in conservation management problems (Bennett et al., unpubl.). When deciding how to spend limited conservation budgets (Buxton et al., 2020), the role of detectability can vastly shape the decision on when to monitor a species and when to take conservation action (Bennett et al., unpubl.; Canessa et al., 2015). Unfortunately, detectability is often unaccounted for in conservation problems, because detection probabilities are not known for a large proportion of taxa (Bennett et al., unpubl.). By not accounting for detectability, wrong decisions about where and when to monitor can often be made, which can be perilous in species at risk of extirpation or extinction (Bennett et al., unpubl.).

\end{document}

\documentclass[12pt]{article}

\usepackage{setspace}
\doublespacing

\usepackage{natbib}
\bibliographystyle{apalike}

\usepackage[margin=1.0in]{geometry}

\usepackage{amsmath}

\usepackage{lineno}

\usepackage{graphicx}
\linenumbers

%opening
\title{Predicting detection probabilities for rare and undersampled North American landbirds}
\author{
	Edwards, Brandon P.M.\\
	\and
	Solymos, Peter\\
	\and
	Stralberg, Diana\\
	\and
	Michel, Nicole\\
	\and
	Robinson, Barry\\
	\and
	Gahbauer, Marcel\\
	\and
	Grinde, Alexis\\
	\and
	Hope, David\\
	\and
	Bennett, Joseph R.\\
	\and
	Smith, Adam C.\\
}

\begin{document}

\maketitle

\begin{abstract}

\end{abstract}

\section{Introduction}

\par Detectability is an important metric for many conservation problems, but often goes unaccounted for, or detectability data just does not exist \citep{bennett_how_2023}.
In conservation management situations, detectability can be used as a metric of sensitivity to evaluate when to monitor a system for a species, or when to act on current information \citep{canessa_when_2015, bennett_when_2018}.
When estimating population sizes of an organism, detectability often comes into play as a way to convert observed counts into densities or estimates of true abundance.
As such, accurate and precise estimates of detectability are important both for making accurate conservation decisions, and estimating accurate population sizes.

\par The wealth of bird data that has been collected and is available in North America means that estimates of detectability for birds are relatively easy to come by \citep{bennett_how_2023}.
The NA-POPS project is an effort to estimate detectability functions that take into account variability in detection through environmental and temporal covariates, for as many of North America's landbird species as possible \citep{edwards_point_2023}. 
To date, the project has estimated availability (the probability a bird sings or gives a cue) and perceptibility (the probability an observer sees or hears a cueing bird) for 319 species of birds, with an additional 19 species having estimates of availability or perceptibility (but not necessarily both) estimated.
This project built off previous efforts by the Boreal Avian Modelling project \citep{cumming_toward_2010} to systematically estimate detectability in landbirds \citep{solymos_calibrating_2013, solymos_evaluating_2018}.
Both the NA-POPS project and previous estimates by Boreal Avian Modelling project use the QPAD methodology, which is a flexible method to estimate availability and perceptibility given a harmonized dataset \citep{barker_ecological_2015, solymos_calibrating_2013}.

\par Despite the efforts of the NA-POPS project, the 338 species of birds considered in the modelling exercise still only account for about 75\% of Partners in Flight's 2016 Landbird Conservation Plan \citep{edwards_point_2023, rosenberg_partners_2016}.
This is because of sample size limitations that are recommended for removal and distance modelling, in that a modeller should have at least 75 data points for reasonable estimates \citep{buckland_introduction_2001}. 
Many of the species that fell short of that threshold in the NA-POPS database are systematically rare or undersampled species, that are either low in population size to begin with, or occur in rare habitats.
Some examples of species not modelled include Kirtland's Warbler (SCIENTIFIC NAME) and Bicknell's Thrush (SCIENTIFIC NAME), both of which have low population sizes, and LeConte's Thrasher (SCIENTIFIC NAME) and Harris's Sparrow (SCIENTIFIC NAME), both of which have rare habitat associations  (PARNTERS IN FLIGHT WATCHLIST CITATION).
Of course, despite not having good estimates of detectability availabile for these rare and undersampled species, these are the groups of species that need these estimates the most as their conservation status often depend on accurate population size estimates, and estimates of trend in those population sizes.

\par Detectability in birds has been shown to at least be partly driven by variation in traits. \cite{johnston_species_2014} showed in a selection of European landbirds that body mass and habitat association had an effect on the detection distance, which influences the perceptibility.
In this case, a larger body size tended to be associated with larger detection distances, which makes sense considering larger body sizes are associated with louder and lower-pitched songs in birds \citep{bowman_adaptive_1979, fletcher_acoustics_1999, ryan_role_1985}.
Habitat associations tends to influence the attenuation of sound, whereby sound diminishes quicker in a closed-forested habitat compared to a more open habitat \citep{waide_tropical_1988, yip_sound_2017}.
A study by \cite{solymos_phylogeny_2018} expanded on this effort by also considering the effects of species traits and phylogeny on both detection distance and cue rate, the latter of which influences the availability of birds.
They found that cue rate is most influenced by phylogeny, and, to a lesser extent, whether a bird was a long-distance migrant or a resident species.
They also found that while the best model that explained detection distance did include phylogeny, the strength of the phylogenetic relationship was reduced to effectively 0 after accounting for traits such as body mass, song pitch, habitat association, and migratory status. 
Both these studies and others (see CITATIONSSSSS) provide useful evidence that detectability in birds can be reasonably predicted by some combinations of phylogeny and traits.

\par One recommendation for future research that come out of \cite{solymos_phylogeny_2018} was to try to use these phylogenetic and trait relationships in detectability to then predict components of detectability (i.e., availability through cue rate, and perceptibility through detection distance) in other birds.
Indeed, by making use of shrinkage factors and partial pooling, such as what can be done in a Bayesian hierarchical models (CITATION), and by accounting phylogenetic relationships and the relative "relatedness" of different species, such as what can be done with Gaussian processes \citep{bernardo_regression_1998, mcelreath_continous_2020}, we can leverage the information provided by species with many data points to use to inform species with much fewer (included potentially zero) data points.
Thus, the goal of this paper is to create a multi-species model of detectability that incorporates species traits and phylogeny, to then predict estimates of detectability for rare and undersampled species of North American landbirds.
To do so, we developed a Bayesian hierarchical version of the QPAD Removal model initially derived in \cite{solymos_calibrating_2013}, that allows for information exchange between species based on phylogenetic distance and binned migratory status.
Phylogenetic distance was coded as a correlation structure and used as a Gaussian random field in this model, and migratory status was coded similar to \cite{solymos_phylogeny_2018} where a species was considered either "migrant" or "resident.
We also developed a Bayesian hierarchical version of the QPAD Distance model initially derived in \cite{solymos_calibrating_2013}, that allows for information exchange between species based only on the traits that came out as being significant in \cite{solymos_phylogeny_2018}.
These traits include migratory status (coded the same as removal models), habitat association (coded as either "open" or "closed"), song pitch, and body mass.
We also developed a Bayesian single species version of the QPAD Removal and Distance models (i.e., as opposed to the maximum likelihood models available in the \texttt{detect} R package; \cite{solymos_detect_2020}) to allow for direct comparison through cross validation.
Finally, we ran these models with the full set of data points used in \cite{edwards_point_2023}, plus additional data for seven species that did not meet the initial sample size thresholds for NA-POPS: LeConte's Thrasher, Bicknell's Thrush, Lesser Prairie-chicken, Harris's Sparrow, Kirtland's Warbler, Tricolored Blackbird, and Spotted Owl. 
All seven of these bird species are listed on Partner's in Flight's watchlist for various threats (CITATION). 

\subsection{Predictions}
\begin{enumerate}
	\item\label{pred1} For both the Bayesian hierarchical QPAD removal model and Bayesian hierarchical QPAD distance model, estimates of cue rate and EDR, respectively, will change very little from those estimated from the original NA-POPS database \cite{edwards_point_2023}. 
	This is because the species that were modelled in \citet{edwards_point_2023} had a sufficient number of data points to stand on their own, and so adding in species-level effects should have little effect on estimates. 
	\item\label{pred2} A multi-species model will increase the precision of the estimates of cue rate and EDR, but only up to a certain sample size. 
	This is because there are many species for which NA-POPS has thousands to even tens of thousands of data points; given precision generally increases with sample size, after a certain sample size, a single-species vs multi-species estimate of cue rate and EDR should be similarly precise. 
	Put differently, we expect that species with a smaller sample size will have precision of cue rate and EDR estimates improved with a multi-species model.
	\item\label{pred3} A multi-species model will have greater predictive capacity than a single-species model, particularly for species which are data-sparse.
\end{enumerate}

\par With the predictions above, we then expect to be able to generate estimates of cue rate and EDR for species with very little data, thus relaxing previous sample size requirements; additionally, we expect to be able to predict cue rate and EDR for species with no data by making use of the shrinkage effect offered by a multi-species model.

\section{Methods}

\par The following subsections outline the modelling techniques used in this study as well as the data that were collected. 
We start by introducing the Bayesian hierarchical QPAD removal and distance models, and then describe the input.
Finally, we describe the experiments conducted to assess model performance and model comparison, which then allow us to make predictions for the seven undersampled species of interest.

\subsection{Bayesian Hierarchical Models}

\par For any point count that employs removal sampling \citep{alldredge_time--detection_2007, farnsworth_removal_2002} and/or distance sampling \citep{buckland_introduction_2001, buckland_distance_2015}, let $Y_{sijk}$ be the observed count of species $s$ during sampling event $i$, occuring in time band $j \in [1,J]$ and/or distance band $k \in [1,K]$.
That is, if we have a point count where removal sampling or distance sampling (but not both) is not used, then we have that either $J = 1$ or $K = 1$, respectively; whereas, if we have a point count where both removal sampling and distance sampling is used, then we have that both $J,K \geq 2$.
The case where $J = K = 1$ is where neither removal sampling nor distance sampling is used, and so these data points are not appropriate for the modelling techniques described here.

\par The following two subsections detail the Bayesian hierarchical QPAD Removal and Distance Models.

\subsubsection{Bayesian Hierarchical QPAD Removal Model}

\par For removal modelling, we are not interested in distance to the bird, and so we will sum counts from each sampling event $i$ over all distance bands. 
Thus, we will be considering counts $Y_{sij.} = \sum_{k=1}^{K}$. 
From \citet{solymos_calibrating_2013}, we model $Y_{sij.}$ as
$$Y_{sij.} \sim multinomial\left(Y_{si..}, \Pi_{si.}\right).$$

\par For sample $i$, $\Pi_{si.}$ is the corresponding probability vector that determines how the total number of species $s$ observed at sample $i$ (i.e., $Y_{si..}$) are to be distributed across the $J$ time bands; in essense, it is the mixing parameter.
Let $t_{ij}$ be the maximum time for time band $j$ during sampling event $i$, and let $\phi_s$ be the unknown cue rate for species $s$.
We then have the following calculation of each component $\pi_{sij}$ of the mixing parameter $\Pi_{si.}$:

\begin{equation*}
	\pi_{sij} = 
	\begin{cases}
		\dfrac{\exp\left\{ -t_{i,j-1}\phi_{s} \right\} - \exp\left\{ -t_{ij}\phi_{s} \right\}}{1 - \exp\left\{ -t_{iJ}\phi_{s} \right\}} & \text{for } j > 1 \\
		1 - \sum_{n = 2}^{J} \pi_{sin} & \text{for } j = 1
	\end{cases}
\end{equation*}

\par We are most interested in estimating $\phi_s$, the cue rate for species $s$. 
To allow for the partial pooling of estimates for each species to be based on a phylogenic relationship, we can use Gaussian processes, which effectively allow for "continuous" factors \citep{bernardo_regression_1998, mcelreath_continous_2020}. 
Let $\Phi$ be the vector of cue rates (of length $s$). 
We then have:

$$\log \Phi \sim MVN\left( \mu_{[s]}, \Sigma_{[s \times s]} \right).$$

\par The covariance function $\Sigma$ (denoted with subscript $[s \times s]$ to show dimensionality) determines the strength of the partial pooling, and can be any function that contains a measure of distance. 
In this case, we will create a function based on phylogenetic distance $D_{[s \times s]}$, i.e.,

$$\Sigma_{[s \times s]} = D_{[s \times s]}\lambda\sigma.$$

\par Here, $\lambda$ is Pagal's Lambda (CITATION), which determines the strength of the phylogenetic relationship. 
We set $\lambda = 0.76$, consistent with previous results \cite{solymos_phylogeny_2018}. 
We also have the extra variance term $\sigma \sim exp(5)$.

\par Finally, the vector of mean cue rate $\mu_{[s]}$ is just a function of the species migratory strategy $\alpha_M$, where the migratory strategy $M$ could either be Resident or Migrant. 
We set priors on both migratory strategies to be $\alpha_M \sim N(-1, 0.01)$, because on the log scale we would expect cue rate to be negative. 

\par Results of simulated priors can be found in Appendix TO DO.

\subsubsection{Bayesian Hierarchical QPAD Distance Model}

\par For distance modelling, we are not interested in the time bin that the bird was recorded in, and so we will sum counts from each sampling event $i$ over all time bands. 
Thus, we will be considering counts $Y_{si.k} = \sum_{j=1}^{J}$. 
From \citet{solymos_calibrating_2013}, we model $Y_{si.k}$ as
$$Y_{si.k} \sim multinomial\left(Y_{si..}, \Pi_{si.}\right).$$

\par For sample $i$, $\Pi_{si.}$ is the corresponding probability vector that determines how the total number of species $s$ observed at sample $i$ (i.e., $Y_{si..}$) are to be distributed across the $K$ distance bins; in essense, it is the mixing parameter. 
Let $r_{ik}$ be the maximum distance for distance bin $k$ during sampling event $i$, and let $\tau_s$ be the unknown effective detection radius for species $s$. 
We then have the following calculation of each component $\pi_{sik}$ of the mixing parameter $\Pi_{si.}$:

%\begin{equation*}
%	\pi_{sik} = 
%	\begin{cases}
%		\dfrac{\left( 1 - \exp\left\{ -\dfrac{r_{i,k}^2}{\exp\left\{\tau_s^2\right\}} \right\} \right) - \left( 1 - \exp\left\{ -\dfrac{r_{i,k-1}^2}{\exp\left\{\tau_s^2\right\}} \right\} \right)}
%		{ 1 - \exp\left\{ -\dfrac{r_{i,K}^2}{\exp\left\{\tau_s^2\right\}} \right\} } & \text{for } k > 1 \\
%		1 - \sum_{n = 2}^{J} \pi_{sin} & \text{for } k = 1
%	\end{cases}
%\end{equation*}

\begin{equation*}
	\pi_{sik} = 
	\begin{cases}
		\dfrac{f(r_{i,k}, \tau_s) - f(r_{i,k-1}, \tau_s)}{f(r_{i,K}, \tau_s)} & \text{for } k > 1 \\
		1 - \sum_{n = 2}^{K} \pi_{sin} & \text{for } k = 1
	\end{cases}
\end{equation*}

where 
$$f(r,\tau) =  1 - \exp\left\{ -\dfrac{r^2}{\exp\left\{\tau^2\right\}} \right\} .$$

\par We are most interested in estimating $\tau_s$, the effective detection radius for species $s$. 
To allow for the partial pooling of estimates for each species to be based on traits, we set up a mixed effects model as follows:
$$\log \tau_s \sim N(\mu_s, \sigma).$$

\par In this model, mean effective detection radius for species $s$ (i.e., $\mu_s$) is a function of the following traits as described in \cite{solymos_phylogeny_2018}: an overall intercept term $\alpha_0$; migratory strategy $\alpha_M$, where $M$ can be Migratory or Resident; habitat preference $\alpha_H$, where $H$ can be Open or Closed habitat; log body size $\beta_b$; and log song pitch $\beta_p$. 
That is, we have
$$ \mu_s = \alpha_0 + \alpha_{M_s} + \alpha_{H_s} + \beta_b \times \log BodySize_s + \beta_p \times \log SongPitch_s$$
$$\alpha_0 \sim N(0.05, 0.1)$$
$$ \alpha_M, \alpha_H \sim N(0, 0.05)$$
$$ \beta_b \sim N(0.01, 0.005)$$
$$ \beta_p \sim N(-0.01, 0.005) $$
$$\sigma \sim exp(5)$$

\par Results of simulated priors can be found in Appendix TO DO.

\subsection{Single Species Models}

\par To directly compare the multi-species models described above to its single-species equivalent in a cross-validation (see below), we built two Bayesian single-species models in Stan.
The single species models used the same likelihood statements as above for the count data, except rather than cue rate and EDR being estimated with shared information, they are estimated independently from each other.
That is, $\log \phi \sim N(0,1) \forall \phi_{s}$, and $\log \tau \sim N(0,1) \forall \tau_s$, thus making it equivalent to the Null model in \citet{edwards_point_2023}, just estimated in a Bayesian framework rather than maximum likelihood. 

\subsection{Data Collection}
\subsubsection{Point Count Data}
\par We used the same standardized database of point count data that was put together through the NA-POPS project \cite{edwards_point_2023}. 
This consists of $712138$ point counts conducted across $292$ projects in Canada and the United States, $422514$ of which used removal sampling and $522820$ of which used distance sampling.

\subsubsection{Phylogenetic Trees}
\par We accessed 2000 pseudo-posterior trees from \citet{jetz_global_2012} that contains phylogeny for all species that contained sufficient data for removal modelling ($>75$ observations as per \cite{edwards_point_2023,solymos_evaluating_2018,buckland_introduction_2001}). 
We also requested that the phylogenies include the following bird species with insufficient data from NA-POPS: LeConte's Thrasher, Bicknell's Thrush, Lesser Prairie-chicken, Harris's Sparrow, Kirtland's Warbler, Tricolored Blackbird, and Spotted Owl. 
All seven of these bird species are listed on Partner's in Flight's watchlist for various threats (CITATION). 
This resulted in 2000 pseudo-posterior trees that contained 323 species.

\par We generated a phylogenetic correlation matrix $D$ using the same methods outlined in the Appendix of \citet{solymos_phylogeny_2018} that generate distances between species, and further processed using the R package "ape" \citet{paradis_ape_2019}.

\subsubsection{Species Traits}
\par Species traits (song pitch, migratory vs resident, open vs. closed habitat) were obtained from Birds of the World accounts of all the birds to model in this study (CITATION). 
Body mass for each bird species was obtained form the Elton traits dataset (CITATION). 
Where body mass was not available through the Elton traits dataset, the Birds of the World account was used instead.

\subsection{Model Assessment}

\par To test prediction \ref{pred1}, we ran the Bayesian hierarchical QPAD Removal and Distance models on the set of species for which there was sufficient data in the original NA-POPS analysis.
In this case, we ran both removal model for 316 species of birds, and we ran the distance model for 317 species of birds (Tricoloured Blackbird was included in the distance models but not the removal models).
We accessed the single-species estimates of cue rate and EDR from the NA-POPS Null models using the \texttt{napops} R package (CITATION).
We then modelled the difference in estimated cue rates and EDRs as
$$\Delta X \sim N(0,\sigma)$$
$$\sigma \sim exp(1)$$

where $\Delta X$ is calculated as the mean multi-species estimate of cue rate or EDR minus the mean single-species estimate of cue rate of EDR.
If prediction \ref{pred1} is true, we would expect to see a distribution of modelled differences centred about 0, indicating no real difference in estimates of cue rate or EDR between single-species and multi-species models, for species for which we have sufficient data.

\par To test prediction \ref{pred2}, we modelled the standard deviations about each estimate of cue rate or EDR from both the multi-species and single-species models, and regressed them against the log sample size per species.
That is
$$SD \sim N(\alpha_M + \beta_M\times \log(N), \sigma)$$
$$\alpha, \beta \sim N(0,1)$$
$$\sigma \sim exp(1)$$

where $\alpha$ is the model-specific intercept and $\beta$ is the model-specific slope.
In both cases, $M=1$ corresponds to the single species model and $M=2$ corresponds to the multi-species model.

\par If prediction \ref{pred2} is true, we would expect to see a higher value of $\alpha_1$ than for $\alpha_2$, corresponding to a higher baseline standard deviation in the single species models.
We would also likely see a slightly smaller (i.e., more negative) value of $\beta_1$ than $\beta_2$ (which should also be negative), indicating that the standard deviation decreases with log sample size for both models, but should decrease at a slower rate with the multi-species model (which has a lower baseline standard deviation to begin with) than the single-species model.

\par To test prediction \ref{pred3}, we ran a k-fold cross-validation with 5 folds, stratified by species. That is, for each cross-validation run, we held out 20\% of the data for each species for both the multi-species and Bayesian single-species models. 
We then estimated multi-species estimates of cue rate and EDR and single-species estimates of cue rate and EDR for each cross-validation run, and used these parameter estimates to calculate the posterior predictive density of each of the held out data points. 
We then calculated the difference in posterior predictive density for removal and distance models, by subtracting the pointwise posterior predictive density of the single-species model from the pointwise posterior predictive density of the multi-species model.
Finally, we took the mean of the difference in pointwise posterior predictive density for each species to create a species-specific mean pointwise predictive density.
With this, positive values of this mean would indicate that the multi-species model was preferred, and negative values would indicate that the single-species model was preferred.
If prediction \ref{pred3} is true, then we would expect to see majority positive values of the specific-specific mean pointwise predictive density.


\subsection{Predicting Detection Probabilities}
\par To predict estimates of cue rate or EDR from the multi-species models, we adopted two possible approaches.
If the NA-POPS database contained some point-count data for the species of interest, we simply added these data into the model.
This is different from \citet{edwards_point_2023, solymos_calibrating_2013, solymos_evaluating_2018}, where any species with less than 75 data points were removed from the analysis.
For species that did not have any data in the NA-POPS database, we generated a row of traits data for the species and modelled the species just as a parameter.
We then ran the multi-species models described above using these additional data points (and hence an updated phylogenetic correlation structure for the removal model) to generate estimates of cue rate and EDR for these new species.

\subsection{Analysis}
\par All analyses were performed in R version TO DO: CHECK (CITATION).
Manipulation of phylogenetic trees, including finding a consensus tree for visualization and generating variance-covariance matrices based on the 2000 bootstrapped trees, was done using the \texttt{ape} R package.
All Bayesian models were coded in the Stan Probabilistic Language \cite{stan_development_team_stan_2019}, and models were run using the \texttt{cmdstanr} R package (CITATION) and visualized using the \texttt{bayesplot} R package (CITATION).
Each of the single-species models and multi-species models running both the initial comparison set of species and the final predictive set of species, as well as the models created to assess performance, were run using 1000 warmup iterations and 2000 sampling iterations, each on 4 chains for a total of 8000 draws per model run.
Cross-validation models were run with 250 warmup iterations and 500 sampling iterations, each on 4 chains for a total of 2000 draws per model run.
All of the models made use of Stan's reduce\_sum functionality, which allows for within-chain parallelization as well as the usual across-chain parallelization (CITATION).
This was particularly useful in our case as the multi-species removal model contained over 3.5 million data points and the multi-species distance model contained over 4.5 million data points, and so computational expense was a large concern.
All models were run on one of two multi-core processing servers, one running Ubuntu 20.04.4 LTS and one running Ubuntu 20.04.6 LTS.


\section{Results}
\subsection{Removal Model Comparisons}

\par We found very little evidence for an overall change in cue rate for a given species modelled by a single-species model vs a multi-species model (Figure \ref{fig:removal_1vs1}A).
Seven out of the 316 species considered for removal modelling had a percent change in cue rate of more than 20\%.
All seven of these species were among the lowest cue rates in both the single-species and multi-species context (Figure \ref{fig:removal_1vs1}B).

\begin{figure}[h!]
	\includegraphics{../output/plots/removal_1vs1.png}
	\caption{TO DO}
	\label{fig:removal_1vs1}
\end{figure}


\par We found much less support for a difference in cue rate between resident and migrant species than was found in \citet{solymos_evaluating_2018}. 
Mean overall log cue rate in the multi-species model was TO DO; both estimated effects for resident and migrant species were effectively 0 (Figure \ref{fig:removal_params}).

\begin{figure}[h!]
	\includegraphics{../output/plots/removal_other_params.png}
	\caption{TODO}
	\label{fig:removal_params}
\end{figure}

\par We found some evidence that a multi-species removal model improved the precision of cue rate estimates compared to estimates from a single-species model.
The intercept of standard deviations of cue rates estimated in a single species model was greater than the intercept of standard deviations of cue rates estimated in a multi-species model, with very little overlap in their 95\% credible intervals (Figure \ref{fig:removal_sd}A).
This implies that the baseline standard deviation of cue rates is higher in a single species model than a multi species model for very low sample sizes.
However, the slope of standard deviations of cue rates estimated in a single species model was farther from 0 (i.e., more negative) than the slope of standard deviations of cue rates estimated in a multi-species model, with very little overlap in their 95\% credible intervals (Figure \ref{fig:removal_sd}B).
This implies that the decrease in standard deviation as sample size increases is lesser in a multi-species model than a single-species model.
This indicates that after a certain sample size, it is likely that the difference in standard deviation between a single species and multi species model would be negligible.

\begin{figure}[h!]
	\includegraphics{../output/plots/removal_sd_plot.png}
	\caption{TODO}
	\label{fig:removal_sd}
\end{figure}

\par The cross validation revealed that single species model actually had better predictive capacity for more species than the multi-species model, but the difference in predictive capacity was overall very small.
Mean overall difference in posterior predictive density was $-2.21 \times 10^{-5}$.
Single-species models tended to have better predicitve capacity when sample size was low (again noting a very small difference), and a negligible difference as sample size increased (Figure \ref{fig:removal_cv}).
There did not appear to be any patterns in model selection among phylogeny or traits (see Appendix SOMETHING for full phylogenetic tree).
Six out of the 7 species that had a percent change in cue rate of more than 20\% had the multi-species model chosen as the better predictive model.
Common Redpoll was the only species that had the single-species model chosen (but please see Discussion about this particular species).

\begin{figure}[h!]
	\includegraphics{../output/plots/cv_removal_diff_vs_n.png}
	\caption{TODO}
	\label{fig:removal_cv}
\end{figure}

\subsection{Distance Model Comparisons}

\par We found some evidence for a change in modelled EDR between single species and multi-species distance models, in that the multi-species model tended to produce slightly smaller estimates of EDR (Figure \ref{fig:distance_1vs1}A). 
Still, the majority of the species had very little change; six out of the 317 species had a percent change in EDR of more than 20\% (Figure \ref{fig:distance_1vs1}B).

\begin{figure}[h!]
	\includegraphics{../output/plots/distance_1vs1.png}
	\caption{TO DO}
	\label{fig:distance_1vs1}
\end{figure}

\par We found little support for migration strategy having any effect on EDR.
There is some evidence that a resident species may have a slightly higher EDR than a migrant species, but the 95\% credible intervals of both effects overlapped with the 95\% credible interval of the overall estimated intercept (Figure \ref{fig:distance_params}A).
We did find similar support for change in EDR by body size and pitch, in that body size tended to have an increasing effect on EDR, and song pitch had a decreasing effect on EDR (Figure \ref{fig:distance_params}B).

\begin{figure}[h!]
	\includegraphics{../output/plots/distance_other_params.png}
	\caption{TODO}
	\label{fig:distance_params}
\end{figure}

\par We found little evidence that a multi-species model improves the precision of EDR compared to that of a single-species model.
The 95\% credible intervals for both estimated slopes and intercepts overlapped with each other, indicating that the standard deviations are effectively the same for a single-species and multi-species estimate as sample size increases (Figure \ref{fig:distance_sd}).

\begin{figure}[h!]
	\includegraphics{../output/plots/distance_sd_plot.png}
	\caption{TODO}
	\label{fig:distance_sd}
\end{figure}

\par The cross validation revealed that the multi-species model tended to have slightly more support than the single-species model in terms of predicting EDR.
However, similar to the removal model cross-validation results, the overall difference in posterior predictive density was very small, with a mean difference of $1.85 \times 10^{-4}$.
Multi-species models tended to have better predicitve capacity when sample size was low (again noting a very small difference), and a negligible difference as sample size increased (Figure \ref{fig:distance_cv}).
Interestingly, only two of the six species with a greater than 20\% change in EDR had the multi-species model chosen as the better predictive model (NOTE this could change as new results come in).
TO DO: Any patterns among traits?

\begin{figure}[h!]
	\includegraphics{../output/plots/cv_distance_diff_vs_n.png}
	\caption{TODO}
	\label{fig:distance_cv}
\end{figure}

\subsection{Predictions of Cue Rate and EDR for Undersampled Species}

\par We used the multi-species removal model and distance model to generate predictions of cue rate and EDR, respectively, for 7 species of undersampled birds from the original NA-POPS analysis (6 for distance modelling because Tricoloured Blackbird had sufficient distance sampling data).

\par The multi-species removal model produced reasonable estimates of cue rates for all seven species of bird considered here (Figure \ref{fig:removal_predictions}).
Species where the sample size requirements were relaxed (i.e., all except Bicknell's Thrush and Kirtland's Warbler) had reasonable precision associated with the estimates.
One exception to this was LeConte's Thrasher, where the precision was rather low.
The species for which we had no data (i.e., Bicknell's Thrush and Kirtland's Warbler) had estimates of cue rate similar to that of their closely-related species, demonstrating the shrinkage effect of the multi-species model.

\begin{figure}[h!]
	\includegraphics{../output/plots/removal_predictions_plot.png}
	\caption{TODO}
	\label{fig:removal_predictions}
\end{figure}

\par The multi-species distance model also produced reasonable estimates of EDR for all six species of landbird considered here (Figure \ref{fig:distance_predictions}).
Species where the sample size requirements were relaxed (i.e., all except Bicknell's Thrush) had fairly precise estimates, except for LeConte's Thrasher and Lesser Prairie Chicken where the estimated EDR spanned greater than 50 m in either direction.
The species for which we had no data (i.e., Bicknell's Thrush) had an estimate of EDR similar to what would be expected by a bird of its size, song pitch, and habitat association, demonstrating the shrinkage effect of the multi-species model.

\begin{figure}[h!]
	\includegraphics{../output/plots/distance_predictions_plot.png}
	\caption{TODO}
	\label{fig:distance_predictions}
\end{figure}

\section{Discussion}

\par In this paper, we developed a multi-species model of detectability by expanding upon the QPAD methodology developed by \citet{solymos_calibrating_2013}.
We applied the multi-species QPAD removal model and multi-species QPAD distance model to the large dataset of point counts collected by the NA-POPS project \citep{edwards_point_2023}, and compared estimates of cue rate and EDR from a multi-species context to estimates of cue rate and EDR from the previous single species context.
Given the results of these comparisons, we sought to predict these components of detectability for seven species of North American landbird that are rare and undersampled.

\subsection{Removal Modelling}
\par Estimates of cue rate changed very little from species that were originally modelled by the single-species models in \citet{edwards_point_2023}, to the same ones modelled in the multi-species model here.
The 5-fold cross validation tended to show either no preference for one model or another when sample size was large, and a preference (albeit an extremely small preference) for single-species models when sample size was smaller.
When sample size was small, a multi-species model tended to produce more precise estimates of cue rate.
Given the extremely small preference for single-species models under a lower sample size, it would appear that the bias-variance tradeoff is likely a welcome tradeoff here, because the decrease in variance is met by only a very slight increase in bias.

\par There were seven species where the relative difference in cue rate between the single-species model and multi-species model was over 20\% (noting that this threshold was chosen arbitrarily).
These species (Willow Ptarmagin, Rusty Blackbird, Red-breasted Sapsucker, Common Redpoll, Calliope Hummingbird, Allen's Hummingbird, and Bohemian Waxwing), mostly had fairly small sample sizes ($\leq 515$ for all species except Red-breasted Sapsucker [$n = 1854$] and Common Redpoll [$n = 2009$]), had the lowest cue rate in the single-species models, and had among the lowest cue rates in the multi-species model.
For removal modelling, all those species except for Common Redpoll ended up having the multi-species model chosen as the more predictive model, indicating that sharing information among related species tended to improve their estimates of cue rate.
Common Redpoll was likely an exception to this because the original estimate of cue rate in NA-POPS was a wild underestimate (cue rate of TO DO), likely because of a convergence issue in the \texttt{detect} package.
It could be worth running Common Redpoll using the Bayesian single species model developed here to see if the estimate of cue rate is still extremely low, or if the vague but still somewhat informative prior helps to bring the estimated parameter into a more sensible space.
It could also be worth investigating the merit of this type of modelling to estimate cue rate for species where the cue rate is extremely low.
In our case, 41 out of the 79 species that had the lowest 25\% quantile of cue rate in the single-species model ended up having the multi-species model chosen as the preferred model.
TO DO Re-read results of Martin-Schwarze.

\par Although this was not a primary goal of our study, we found much less support than \citet{solymos_phylogeny_2018} for migration strategy having an effect on cue rate.
Both the estimate effects for resident species and migratory species were effectively 0, with very little of the density above or below 0.025 of 0.
One potential reason is that we had proportionally more resident species in our analysis (85 out of 316; 26.9\%) compared to \citet{solymos_phylogeny_2018} (23 out of 141; 16.3\%), and so the richer information provided by these additional species could have brought the effect down to 0.
Additionally, while \citet{solymos_phylogeny_2018} only regressed estimates of cue rate against the predictor variables, we actually estimated cue rate within the context of the same predictors, and so the phylogenetic effect could have just overwhelmed the estimates, making the migratory strategy meaningless.

\subsection{Distance Modelling}
\par Estimates of EDR also changed very little from species that were originally modelled by the single-species models in \citet{edwards_point_2023}, to the same ones modelled in the multi-species model here.
However, there was evidence for an average small downward shift in EDRs when modelled under a multi-species context.
Still, the 5-fold cross validation tended to show little to no preference for one more or another when the sample size was large, and a slight (but still very small) preference for the multi-species model when the sample size was smaller.
There was less evidence for the multi-species model being able to improve the precision of EDR estimates across sample sizes, indicating that a single-species model may mostly be sufficient for precise estimates.
Under this context, it would appear that a multi-species model may slightly bring down estimates of EDR for species with a smaller sample size.

\par Talk about species with a large relative difference

\par Talk about parameters

\subsection{Predictions of Cue Rate and EDR}

* talk about Bicknell's Thrush removal and distance modelling

\bibliography{refs}

\end{document}
